\documentclass[12pt]{article}%
\usepackage{amsfonts}
\usepackage{fancyhdr}
\usepackage{comment}
\usepackage[a4paper, top=2.5cm, bottom=2.5cm, left=2.2cm, right=2.2cm]%
{geometry}
\usepackage{times}
\usepackage{amsmath}
\usepackage{changepage}
\usepackage{stfloats}
\usepackage{amssymb}
\usepackage{graphicx}
\usepackage{indentfirst}
\setlength{\parindent}{2em}
\setcounter{MaxMatrixCols}{30}
\newtheorem{theorem}{Theorem}
\newtheorem{acknowledgement}[theorem]{Acknowledgement}
\newtheorem{algorithm}[theorem]{Algorithm}
\newtheorem{axiom}{Axiom}
\newtheorem{case}[theorem]{Case}
\newtheorem{claim}[theorem]{Claim}
\newtheorem{conclusion}[theorem]{Conclusion}
\newtheorem{condition}[theorem]{Condition}
\newtheorem{conjecture}[theorem]{Conjecture}
\newtheorem{corollary}[theorem]{Corollary}
\newtheorem{criterion}[theorem]{Criterion}
\newtheorem{definition}[theorem]{Definition}
\newtheorem{example}[theorem]{Example}
\newtheorem{exercise}[theorem]{Exercise}
\newtheorem{lemma}[theorem]{Lemma}
\newtheorem{notation}[theorem]{Notation}
\newtheorem{problem}[theorem]{Problem}
\newtheorem{proposition}[theorem]{Proposition}
\newtheorem{remark}[theorem]{Remark}
\newtheorem{solution}[theorem]{Solution}
\newtheorem{summary}[theorem]{Summary}
\newenvironment{proof}[1][Proof]{\textbf{#1.} }{\ \rule{0.5em}{0.5em}}

\usepackage{mathtools}

\newcommand{\Q}{\mathbb{Q}}
\newcommand{\R}{\mathbb{R}}
\newcommand{\C}{\mathbb{C}}
\newcommand{\Z}{\mathbb{Z}}

\begin{document}

\title{Investigation on income inequality of the Islands\\
\Large STAT3003 Midterm A - Survey Plan}
\author{ZHENG Weijia (William, 1155124322)}
\date{April 6, 2021}
\maketitle



\section{Introduction and Notations}
Income inequality continues to receive world-wide attention. The most famous method to measure it to use the Gini Index proposed by Gini. 
Gini index gives a number inside the interval $[0,1]$, the more the index close to 1, the worse the inequality problem is.

Assume we will investigate the situation of income inequality for the scale of a town (e.g., Hofn, Vardo and Takazaki are all towns).

Then denote the population of the town as $n$, note that this is given.

Denote the total amount of wealth of the town as $M$, which is also given by consulting the Hall of the town.

Denote the average amount of wealth as $\mu = \frac{M}{n}$.

For the i-th individual (i ranges from 1 to n), denote the his individual amount of wealth as $m_i$, note that $\sum_{i=1}^{n}m_i=M.$

Then if we follow the definition of Gini index, we need to calculate $$G=\frac{1}{2n^2\mu}\sum_{j=1}^{n}\sum_{i=1}^{n}|m_i-m_j|.$$
But this is troublesome because it is hard to estimate the $G$ of a town properly from a sample of smaller size. 
Hence we switch to another method of measuring the inequality.


\section{Mathematical Model}
As said above, hindered by the complex way to calculate the Gini index out and noted that the Islands provide the information of the total amount of a town and every individual's. 
We can then understand the concept of "income euqality" from the respective of the difference between "every person's ability (as a possibility) to gain wealth".

For the i-th person, we regard the ratio between his own wealth and the total wealth, i.e., $p_i=\frac{m_i}{M}$ of the town as his "possibility of gaining a unit of wealth".

Note that $\sum_{i=1}^{n}p_i=1,$ hence the $p_i$ forms a proper probability distribution. Note that the entropy $$H(p_1,\dots,p_n)=-\sum_{i=1}^{n}p_i\log_{n}{p_i}$$ actually measures the 
inequality of the town economy. Hence define it as the "$H$ index" of the economy.

When the town economy is more fair, i.e., everyone has the same probability of earning money: $\frac{1}{n}$, then $$H(p_1,\dots, p_n)=-\sum_{i=1}^{n}p_i\log_{n}{p_i}=-\sum_{i=1}^{n}\frac{1}{n}\log_{n}{\frac{1}{n}}=1.$$ 
When the economy is extremly inequal, i.e., all the money go to one person, then the entropy will be 0.

The following is a simple example of the application of this Mathematical model. Consider a economy with only 3 people, each income is 10, 20, 15. 
And another economy with 5 people with each income  3, 20, 16, 40. The first economy has $H_1=0.9656$ and the second economy with $H_2=0.8223,$ which conforms our intuition.


Note that the $H$ index is a $\tau$-type metric, so all the knowledge for estimating $\tau$ can be applied properly.


\section{Sampling Method}



\section{Preliminary Survey Result}



\end{document}
