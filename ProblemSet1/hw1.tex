\documentclass[12pt]{article}%
\usepackage{amsfonts}
\usepackage{fancyhdr}
\usepackage{comment}
\usepackage[a4paper, top=2.5cm, bottom=2.5cm, left=2.2cm, right=2.2cm]%
{geometry}
\usepackage{times}
\usepackage{amsmath}
\usepackage{changepage}
\usepackage{stfloats}
\usepackage{amssymb}
\usepackage{graphicx}
\usepackage{indentfirst}
\setlength{\parindent}{2em}
\setcounter{MaxMatrixCols}{30}
\newtheorem{theorem}{Theorem}
\newtheorem{acknowledgement}[theorem]{Acknowledgement}
\newtheorem{algorithm}[theorem]{Algorithm}
\newtheorem{axiom}{Axiom}
\newtheorem{case}[theorem]{Case}
\newtheorem{claim}[theorem]{Claim}
\newtheorem{conclusion}[theorem]{Conclusion}
\newtheorem{condition}[theorem]{Condition}
\newtheorem{conjecture}[theorem]{Conjecture}
\newtheorem{corollary}[theorem]{Corollary}
\newtheorem{criterion}[theorem]{Criterion}
\newtheorem{definition}[theorem]{Definition}
\newtheorem{example}[theorem]{Example}
\newtheorem{exercise}[theorem]{Exercise}
\newtheorem{lemma}[theorem]{Lemma}
\newtheorem{notation}[theorem]{Notation}
\newtheorem{problem}[theorem]{Problem}
\newtheorem{proposition}[theorem]{Proposition}
\newtheorem{remark}[theorem]{Remark}
\newtheorem{solution}[theorem]{Solution}
\newtheorem{summary}[theorem]{Summary}
\newenvironment{proof}[1][Proof]{\textbf{#1.} }{\ \rule{0.5em}{0.5em}}

\usepackage{mathtools}

\newcommand{\Q}{\mathbb{Q}}
\newcommand{\R}{\mathbb{R}}
\newcommand{\C}{\mathbb{C}}
\newcommand{\Z}{\mathbb{Z}}

\begin{document}

\title{STAT3003 Problem Sheet 1}
\author{ZHENG Weijia (William, 1155124322)}
\date{\today}
\maketitle



\section{Q1}

We want to prove $E[\hat{\sigma}^2]=\frac{N}{N-1}\sigma^2.$

Note that $\sigma^2=\frac{1}{N}\sum_{j=1}^{N}(u_j-\mu)=\frac{1}{N^2}[(N-1)\sum_{j=1}^{N}u_j^2-\sum_{j=1}^{N}\sum_{k\neq j}u_ju_k]$

Hence R.H.S.=$\frac{N}{N-1}\sigma^2=\frac{1}{N(N-1)}[\sum_{j=1}^{N}\sum_{k\neq j}u_j^2-\sum_{j=1}^{N}\sum_{k\neq j}u_ju_k].$

Also note that $\hat{\sigma}^2=\frac{1}{n-1}\sum_{i=1}^{n}(Y_i-\bar{Y})^2=\frac{1}{n(n-1)}[(n-1)\sum_{i=1}^{n}Y_i^2-\sum_{j=1}^{n}\sum_{k\neq j}Y_jY_k]$

$=\frac{1}{n(n-1)}[(n-1)\sum_{i=1}^{n}u_i^2Z_i^2-\sum_{j=1}^{n}\sum_{k\neq j}u_ju_kZ_jZ_k].$

Hence L.H.S.=$E[\hat{\sigma}^2]=\frac{1}{n}\sum_{i=1}^{n}u_i^2E[Z_i^2]-\frac{1}{n(n-1)}\sum_{j=1}^{n}\sum_{k\neq j}u_ju_kE[Z_jZ_k]$

$=\frac{1}{n}\sum_{i=1}^{n}u_i^2E[Z_i^2]-\frac{1}{n(n-1)}\sum_{j=1}^{n}\sum_{k\neq j}u_ju_kE[Z_jZ_k]$

$=\frac{1}{n}\sum_{i=1}^{n}u_i^2\frac{n}{N}-\frac{1}{n(n-1)}\sum_{j=1}^{n}\sum_{k\neq j}u_ju_k\frac{n(n-1)}{N(N-1)}$

$=\frac{1}{N}\sum_{i=1}^{n}u_i^2-\sum_{j=1}^{n}\sum_{k\neq j}u_ju_k\frac{1}{N(N-1)}=\frac{1}{N(N-1)}[\sum_{j=1}^{N}\sum_{k\neq j}u_j^2-\sum_{j=1}^{N}\sum_{k\neq j}u_ju_k]$=R.H.S..

Q.E.D.

~\ 

\section{Q2}
Population: all worms in the field. 

Sampling units: do partition the field into say, $n$, disjoint parts with equal size. We take $k\leq n$ of them. And all worms in each chosen part is a sampling unit.

Frame: the process stated above is to construct a frame, i.e., the frame is consisted of those $k$ sampling units.

Perform SRS: we can label each partition with $i,1\leq i\leq n.$ We can use random numbers in Excel to select $k$ from the $n$.

No, the size does not matter. Because the probability of elements in a larger area or a smaller area have the same probability of being selected.

We need to consider the cost, because investigation into a larger sample cost more naturally.

~\ 

\section{Q3}
Note that $\hat{p}=\frac{1}{n}\sum_{i=1}^{n}Y_i=\frac{430}{1000}=0.43.$ $n=1000,N=9900$

Also, $1-\alpha=95\%$, hence $\alpha=0.05.$ $t_{999,0.975}=1.96.$

Note that both $n\hat{p},n(1-\hat{p}) >5.$

To calculate the 95\% confidence interval for p. It should be $$\hat{p}\pm\sqrt{(1-\frac{n}{N})\frac{1}{n-1}\hat{p}(1-\hat{p})}t_{999,0.975}=[0.401,0.459].$$

~\   

\section{Q4}
By the formulae in the lecture note, we need $$n=\frac{1}{\frac{1}{N}+\frac{d^2}{p(1-p)z_{1-\frac{\alpha}{2}}^2}(1-\frac{1}{N})}$$

With $N=9900,d=0.02$ and $\alpha=0.05, z_{1-\frac{\alpha}{2}}=1.96.$ For the $p$, we utilize the result from last question, plug $p$ with 0.43.

Hence $$n=\frac{1}{\frac{1}{9900}+\frac{0.02^2}{0.43(1-0.43)1.96^2}(1-\frac{1}{9900})}=1901.9.$$ Hence 1902 can be a proper sample size.

~\ 

\section{Q5}

Denote the estimated total as $\hat{\tau}.$ Hence the 95\% C.I. for $\tau$ is $$(\hat{\tau}\pm \sqrt{N^2(1-\frac{n}{N})\frac{s^2}{n}}t_{n-1,1-\frac{\alpha}{2}}).$$

With $N=1500,n=100,s^2=136,\bar{y}=22.5,$ we can first derive that $$\hat{\tau}=N\bar{y}=1500\cdot 22.5=33750.$$

Hence the interval should be $$(33750\pm \sqrt{1500^2(1-\frac{100}{1500})\frac{136}{100}}~1.9842~)=(33750\pm 3353.2393).$$

~ \ 

\section{Q6}

Note that we have a pre-sample estimate for $\sigma$, which is $\sqrt{s^2}=11.66.$

Hence the proper sample size can be $$n=\frac{1}{\frac{1}{1500}+\frac{1500^2}{1500^2s^2t_{n-1,1-\frac{\alpha}{2}}^2}}.$$

Consult the table we found that when $n=565,$ R.H.S.=565.8022 and when $n=566,$ R.H.S.=565.7979, hence we choose n=566.



~\ 

\section{Q7}

Note that the sample mean $\bar{y}=2$, $s^2=\frac{1}{n-1}\sum_{i=1}^{n}(y_i-\bar{y})^2=\frac{20}{9},$ hence $s=1.49.$

Note that $\alpha=1-0.95=0.05.$ And the the interval should be $$\bar{y}\pm t_{9,0.975}\sqrt{\frac{N-n}{N}}\frac{s}{\sqrt{n}}=2\pm 2.2622\cdot \sqrt{\frac{100-10}{100}}\frac{1.49}{\sqrt{10}}=2\pm 1.0112=[1,3].$$

\end{document}
