\documentclass[12pt]{article}%
\usepackage{amsfonts}
\usepackage{fancyhdr}
\usepackage{comment}
\usepackage[a4paper, top=2.5cm, bottom=2.5cm, left=2.2cm, right=2.2cm]%
{geometry}
\usepackage{times}
\usepackage{amsmath}
\usepackage{changepage}
\usepackage{stfloats}
\usepackage{amssymb}
\usepackage{graphicx}
\usepackage{indentfirst}
\setlength{\parindent}{2em}
\setcounter{MaxMatrixCols}{30}
\newtheorem{theorem}{Theorem}
\newtheorem{acknowledgement}[theorem]{Acknowledgement}
\newtheorem{algorithm}[theorem]{Algorithm}
\newtheorem{axiom}{Axiom}
\newtheorem{case}[theorem]{Case}
\newtheorem{claim}[theorem]{Claim}
\newtheorem{conclusion}[theorem]{Conclusion}
\newtheorem{condition}[theorem]{Condition}
\newtheorem{conjecture}[theorem]{Conjecture}
\newtheorem{corollary}[theorem]{Corollary}
\newtheorem{criterion}[theorem]{Criterion}
\newtheorem{definition}[theorem]{Definition}
\newtheorem{example}[theorem]{Example}
\newtheorem{exercise}[theorem]{Exercise}
\newtheorem{lemma}[theorem]{Lemma}
\newtheorem{notation}[theorem]{Notation}
\newtheorem{problem}[theorem]{Problem}
\newtheorem{proposition}[theorem]{Proposition}
\newtheorem{remark}[theorem]{Remark}
\newtheorem{solution}[theorem]{Solution}
\newtheorem{summary}[theorem]{Summary}
\newenvironment{proof}[1][Proof]{\textbf{#1.} }{\ \rule{0.5em}{0.5em}}

\usepackage{mathtools}

\newcommand{\Q}{\mathbb{Q}}
\newcommand{\R}{\mathbb{R}}
\newcommand{\C}{\mathbb{C}}
\newcommand{\Z}{\mathbb{Z}}

\begin{document}

\title{STAT3003 Problem Sheet 2}
\author{ZHENG Weijia (William, 1155124322)}
\date{\today}
\maketitle



\section{Q1}

Note that $L=4$, $N=N_1+N_2+N_3+N_4 = 225,$ with $N_1=64, N_2=43, N_3 = 92, N_4=26.$ And $\hat{p_1}=\frac{2}{7}, \hat{p_2}=\frac{1}{3}, \hat{p_3}=\frac{8}{21}, \hat{p_4}=\frac{1}{3}.$

Hence we can have the $$\hat{p_{st}}=\frac{1}{N}\sum_{i=1}^{L}N_i \hat{p_i}=0.3393.$$

Also, by $$\hat{Var}(\hat{p_{st}})=\frac{1}{N^2}\sum_{i=1}^{L}N_i^2(1-\frac{n_i}{N_i})\frac{1}{n_i-1}\hat{p_i}(1-\hat{p_i})=\frac{197.979}{225^2}=3.7724*10^{-3}.$$

Therefore, $\sqrt{\hat{Var}(\hat{p_{st}})}=0.0614.$

Our goal is to find $$(~\hat{p_{st}} \pm t_{df,1-\frac{\alpha}{2}}*\sqrt{\hat{Var}(\hat{p_{st}})}~~).$$

By Satterthwaite's Approximation, we have $$df \approx \frac{(\sum_{i=1}^{L}k_i s_{i}^2)^2}{\sum_{i=1}^{L}\frac{(k_i s_i^2)^2}{n_i-1}}, k_i=\frac{N_i(N_i-n_i)}{N^2n_i},$$ and hence $df=45.$

Then $t_{df,1-\frac{\alpha}{2}}=t_{45,0.975}=2.0141.$ Therefore, then result should be $$(~0.3393\pm 2.0141*0.0614~)$$

Done.

\section{Q2}
Note that $L=3, c = 500, c_0 =0.$

Using $$n_i=\frac{(c-c_0)\frac{N_i S_i}{\sqrt{c_i}}}{\sum_{i=1}^{L}N_i S_i \sqrt{c_i}}.$$ 

Note that we have the $\sigma_i$ are given, hence by $\hat{\sigma_i}^2=\frac{N_i}{N_i-1}\sigma_i^2.$ Then we have $s_i = \hat{\sigma_i}.$ With $s_1=1.5067, s_2 = 1.8134, s_3=1.8235.$

Which gives that $n_1=18.1510, n_2=7.9581, n_3=3.8247.$ By doing rounding, we have $n_1=18, n_2=8, n_3=4.$ Under this case, we have the cost to be $506.$

For saving the cost, we need to reduce the sample size in North America by 1. 

Then we have $n_1=17, n_2=8, n_3=4.$ Under this case, the cost is $497,$ which is okay. 

And by $$Var(\bar{Y_{st}})=\frac{1}{N^2}\sum_{i=1}^{L}N_i^2\frac{\sigma_i^2}{n_i}(\frac{N_i-n_i}{N_i-1})=0.08825<0.1.$$

Which can satisfy the corporation's requirement. Hence they can be happy.

Done. 

\section{Q3}
Recall that the function we want to optimize is $$f(n_1,n_2,...,n_L)=\frac{1}{N^2}\sum_{i=1}^{L}N_i^2\frac{\sigma_{i}^2}{n_i}(\frac{N_i-n_i}{N_i-1}).$$

And the constraint is $$g(n_1,n_2,...,n_L)=c-(c_0+c_1n_1+...+c_Ln_L)=0.$$

The lagrangian is $$L=f-\lambda g.$$ 

Note that $$\frac{\partial f}{\partial n_j}=\frac{1}{N^2}\frac{\partial (N_j^3\ \sigma_{j}^2)}{n_j (N_j-1)}=-\frac{N_j^3 \sigma_j^2}{N^2 (N_j-1)n_j^2}.$$

And $$\frac{\partial g}{\partial n_j}=-c_j.$$ 

Therefore we can have $$\lambda c_j = \frac{1}{n_j^2}\frac{N_j^3 \sigma_j^2}{N^2(N_j-1)}.$$ 

Which implies $$n_j=\frac{1}{\sqrt{\lambda}} \sqrt{\frac{N_j}{N_j -1}} \frac{N_j \sigma_j}{\sqrt{c_j} N}.$$

Recall the constraint g=0. We can have $$\sum_{j=1}^{L}c_jn_j=(c-c_0)=\sum_{j=1}^{L} \frac{1}{\sqrt{\lambda}} \sqrt{\frac{N_j}{N_j -1}} \frac{N_j \sigma_j}{\sqrt{c_j} N}.$$

Hence $$\frac{1}{\sqrt{\lambda}}=\frac{c-c_0}{\sum_{j=1}^{L}\sqrt{\frac{N_j}{N_j -1}} \frac{N_j \sigma_j}{\sqrt{c_j} N} } .$$

And therefore, $$n_j=\frac{c-c_0}{\sum_{j=1}^{L}\sqrt{\frac{N_j}{N_j -1}} \frac{N_j \sigma_j}{\sqrt{c_j} N} } \sqrt{\frac{N_j}{N_j -1}} \frac{N_j \sigma_j}{\sqrt{c_j} N}=\frac{(c-c_0)(\sqrt{\frac{N_i}{N_i-1}}\frac{N_i \sigma_i}{\sqrt{c_i}})}{\sum_{i=1}^{L} \sqrt{\frac{N_i}{N_i-1}}N_i \sigma_i \sqrt{c_i} }.$$ Which is want we want. 

Done.

\section{Q4}
By the formulae $\hat{\mu} = \frac{\hat{\tau}}{M},$ we can investigate the $\hat{\tau}$ and then go from $\hat{\tau}$ to get $\hat{\mu}.$

Note that $M=3500, N=108, n=25.$ 

Note that $$\hat{\tau}=N\frac{1}{n}\sum_{i=1}^{n}Y_i=151014.24.$$

And $$t_{n-1,1-\frac{\alpha}{2}}=t_{24,0.975}=2.06.$$

Also, we have $$\hat{Var}(\hat{\tau})=\hat{Var}(N\bar{Y})=N^2(1-\frac{n}{N})\frac{\hat{\sigma_c}^2}{n}.$$

Where the $\hat{\sigma_c}^2$ is the sample variance of cluster totals. Hence $\hat{\sigma_c}^2=149422.2016.$ And $$\hat{Var}(\hat{\tau})=53576824.61.$$

Therefore $\sqrt{\hat{Var}(\hat{\tau})}=7319.6192.$ Hence the 95\% CI for $\tau$ is $$(~151014.24 \pm 2.06*7319.6192~)$$. Therefore 95\% CI for $\mu$ is $(~43,147 \pm 4.3081~).$


\section{Q5}
Note that $\hat{\mu}=\frac{N}{M}\bar{Y}.$ Hence $$Var(\hat{\mu})=Var(\bar{Y})\frac{N^2}{M^2}=\frac{N^2}{M^2}\frac{N-n}{N-1}\frac{1}{n}\sigma_c^2.$$

Hence the width $2d=2t_{n-1,1-\frac{\alpha}{2}}\sqrt{Var(\hat{\mu})} \leq 4.$ Solving this inequality, we then have $$n \geq \frac{1}{\frac{1}{N} + \frac{4(N-1)M^2}{N^3 \sigma_c^2 t_{n-1,1-\frac{\alpha}{2}}} }.$$ 

Now we plug in $M=3500,$ and $\sigma_c^2$ can be used by $149422.2016$ and N be pluged in by $108$. Then we can see the minimal n such that the above inequality hold is $n=44$.

Done. 


\section{Q6}


\end{document}
